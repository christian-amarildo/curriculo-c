\documentclass[]{mcdowellcv}

% For mathematical symbols
\usepackage{amsmath}
\usepackage{hyperref}
% Set applicant's personal data for header
\name{Christian Amarildo}
\address{\href{https://www.linkedin.com/in/christian-amarildo}{LinkedIn/christian-amarildo} \linebreak \href{https://github.com/christian-amarildo}{Github/christian-amarildo}}
\contacts{(+55) 91 99139-5646 \linebreak christianamarildo@hotmail.com}

\begin{document}

% Print the header
\makeheader

% Print the content
\begin{cvsection}{Sumário}
    \vspace{\baselineskip}
    Estudante com 2 anos de estudo na área, atualmente atuando no Laboratório de Cibersegurança (LabSec) da UFPA. Tenho experiência e habilidades em desenvolvimento web e mobile, com conhecimentos sólidos em HTML, CSS, JavaScript, Dart, Flutter e Python. Sou um participante ativo de hackathons e CTFs, sempre buscando aprimorar minhas habilidades em desenvolvimento web e de aplicativos. Sou proativo, curioso e determinado a aprender novas tecnologias para aprimorar minha carreira nessa área.
     \vspace{\baselineskip}
    \begin{itemize}
        \item +150 exercícios concluídos na linguagem Python.
        \item Criação de 5 sites utilizando HTML, CSS e JavaScript.
        \item Conclusão bem-sucedida de 62 exercícios em Dart.
        \item Desenvolvimento de 3 aplicativos mobile com a tecnologia Flutter.
    \end{itemize}
\end{cvsection}


\begin{cvsection}{Experiência}
	\begin{cvsubsection}{LabSec}{Analista em Cibersegurança, Voluntário}{março 2023 - presente}
		\begin{itemize}
                \vspace{\baselineskip}
                \item Criação de sites seguros, utilizando práticas e tecnologias avançadas para proteger informações confidenciais.
			\item Experiência em desenvolvimento web, implementando algoritmos de criptografia para garantir a segurança de dados sensíveis.
			\item Avaliação e recomendação de melhores práticas de segurança em processos de desenvolvimento de software.
			\item Gerenciamento de projetos e auxilio em projetos importantes da faculdade, como segurança em votações.
		\end{itemize}
	\end{cvsubsection}
\end{cvsection}

\begin{cvsection}{Educação}
	\begin{cvsubsection}{Universidade Federal do Pará}{Bacharelado em Ciência da Computação}{março 2023 - presente}
		\begin{itemize}
                \vspace{\baselineskip}
			\item conhecimentos fundamentais em programação, algoritmos e hardware .Desenvolvimento das habilidades analíticas e criativas para resolver 
            problemas complexos de forma eficiente. metodologias de gestão, planejamento, análise de riscos, alocação de recursos e desenvolvimento da oratória
		\end{itemize}
	\end{cvsubsection}
	\begin{cvsubsection}{Anhanguera}{Tecnólogo em Cybersegurança}{março 2023 - presente}
		\begin{itemize}
                \item fundamentos completos da base sobre computação, redes e programação
			\item aprendizado de técnicas avançadas de segurança, utilização de ferramentas e tecnologias essenciais, gestão de incidentes, implementação de políticas de segurança e proteção contra ameaças sofisticadas
		\end{itemize}
	\end{cvsubsection}
\end{cvsection}

\begin{cvsection}{Skills}
	\begin{cvsubsection}{}{}{}
		\begin{itemize}
			\item linguagens de programação: Python, HTML/CSS, JavaScript, dart, Git.
			\item conhecimentos adicionais: flutter,Kali Linux, ZAP Proxy, Nmap, Sqlmap, Capture the Flag, Metodologias OWASP.
                \item ingles intermediário, espanhol básico, pacote office intermediário
		\end{itemize}
	\end{cvsubsection}
\end{cvsection}
\begin{cvsection}{cursos,certificações e habilidades adicionais}
	\begin{cvsubsection}{}{}{}
		\begin{itemize}
                \item preparatório para a certificação security+ e pentest+
			\item Participação em eventos de CTF e desafios/lab online de pentest.
                \item curso de inteligência emocional, programação neurolinguística, empreendedorismo e oratória
		\end{itemize}
	\end{cvsubsection}
\end{cvsection}

\end{document}
