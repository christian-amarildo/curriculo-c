\documentclass[]{mcdowellcv}

% For mathematical symbols
\usepackage{amsmath}
\usepackage{hyperref}
% Set applicant's personal data for header
\name{Christian Amarildo}
\address{\href{https://www.linkedin.com/in/christian-amarildo}{LinkedIn/christian-amarildo} \linebreak \href{https://github.com/christian-amarildo}{Github/christian-amarildo}}
\contacts{(+55) 91 99139-5646 \linebreak christianamarildo@hotmail.com}

\begin{document}

% Print the header
\makeheader

% Print the content
\begin{cvsection}{Summary}
        \vspace{\baselineskip}
	\item Com 2 anos de experiência em cibersegurança e atuação no Laboratório de Cibersegurança (LabSec), tive um impacto significativo na identificação e mitigação de vulnerabilidades em sistemas, fortalecendo a segurança e protegendo informações confidenciais. 
        \item Participei de competições de capture the flag, aprimorando minhas habilidades técnicas e estratégicas.Tenho conhecimento em metodologias OWASP e sou proativo e curioso. Estou determinado a fortalecer as defesas contra ameaças cibernéticas
\end{cvsection}

\begin{cvsection}{Experience}
	\begin{cvsubsection}{LabSec}{Analista em Cibersegurança, Voluntário}{março 2023 - presente}
		\begin{itemize}
                \vspace{\baselineskip}
                \item Criação e implementação de algoritmos de criptografia para garantir a segurança de dados sensíveis.
			\item Desenvolvimento de sites seguros, utilizando práticas e tecnologias avançadas para proteger informações confidenciais.
			\item Análise e fortificação de sistemas contra vulnerabilidades comuns, como ataques de injeção de SQL e cross-site scripting (XSS).
			\item Implementação de certificados SSL/TLS para garantir a segurança das comunicações online.
			\item Avaliação e recomendação de melhores práticas de segurança em processos de desenvolvimento de software.
			\item Auxilio em projetos importantes da faculdade, como segurança em votações.
		\end{itemize}
	\end{cvsubsection}
\end{cvsection}

\begin{cvsection}{Education}
	\begin{cvsubsection}{Universidade Federal do Pará}{Bacharelado em Ciência da Computação}{março 2023 - presente}
		\begin{itemize}
                \vspace{\baselineskip}
			\item conhecimentos fundamentais em programação, algoritmos e inteligência artificial.Desenvolvimento das habilidades analíticas e criativas para resolver problemas complexos de forma eficiente.
                \item segurança da informação, ética digital e responsabilidade social, preparando-me para enfrentar os desafios do mundo digital em constante evolução.
		\end{itemize}
	\end{cvsubsection}
	\begin{cvsubsection}{Anhanguera}{Tecnólogo em Cybersegurança}{março 2023 - presente}
		\begin{itemize}
			\item aprendizado de técnicas avançadas de segurança, utilização de ferramentas e tecnologias essenciais, gestão de incidentes, implementação de políticas de segurança e proteção contra ameaças sofisticadas
		\end{itemize}
	\end{cvsubsection}
\end{cvsection}

\begin{cvsection}{Skills}
	\begin{cvsubsection}{}{}{}
		\begin{itemize}
			\item Programming Languages: Python, HTML/CSS, JavaScript, Node.js, PostgreSQL, Git.
			\item Additional Skills: Kali Linux, ZAP Proxy, Nmap, Sqlmap, Capture the Flag, Metodologias OWASP.
		\end{itemize}
	\end{cvsubsection}
\end{cvsection}
\begin{cvsection}{adicionais}
	\begin{cvsubsection}{}{}{}
		\begin{itemize}
                \item fazendo preparatório para a certificação pentest+
			\item Participação em eventos de CTF e desafios/lab online de pentest.
                \item Conhecimentos em ferramentas de segurança como Burp, Shell Script.
                \item Noções de desenvolvimento seguro.
                \item Familiaridade com Metodologias OWASP, CVSS, CWE e CVE.
		\end{itemize}
	\end{cvsubsection}
\end{cvsection}

\end{document}
